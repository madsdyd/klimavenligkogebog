
\documentclass[11pt, a4paper]{article}
% Set up our encodings.
% The order of the next two matters.
% \usepackage[T1]{fontenc}
\usepackage[utf8]{inputenc}
\usepackage{textcomp}

\usepackage{xcolor}
\usepackage{geometry}
\usepackage{amsmath, amssymb}
\usepackage{graphicx}

\usepackage{xcookybooky}
\usepackage{graphicx}
\usepackage{longtable}
%\usepackage[export]{adjustbox}
\usepackage{enumitem}

%\usepackage{ pbsi }

\usepackage{float}
\usepackage{color}


%\definecolor{magenta}{RGB}{195, 0, 122}
\definecolor{magenta}{RGB}{70, 169, 0}
\color{black}
\usepackage[danish]{babel}
\geometry{
	verbose,
	tmargin=2.5cm,
	bmargin=2.5cm,
	lmargin=2.5cm,
	rmargin=2.5cm
}
\setlength{\parindent}{0ex}

\newcommand{\coo}{CO$_2$ }

\usepackage{hyperref}    % must be the last package
\hypersetup{%
	pdfauthor            = {Miljørådet FPR},
	pdftitle             = {Klimavenlig kogebog},
	pdfsubject           = {Madlavning},
	pdfkeywords          = {madlavning, opskrifter, klima},
	pdfstartview         = {FitV},
	pdfview              = {FitH},
	pdfpagemode          = {UseNone}, % Options; UseNone, UseOutlines
	bookmarksopen        = {true},
	pdfpagetransition    = {Glitter},
	colorlinks           = {true},
	linkcolor            = {black},
	urlcolor             = {magenta},
	citecolor            = {black},
	filecolor            = {black},
}

%%%%%%%%%%%%%%%%%%%%%%%%%%%%%%%%%%%%%%%%%%%%%%%%%%%%%%%%%%%%%%%%%%%%%%%%%%%%%%%%%%%%%%
% Use dejavu sans font
%\usepackage{DejaVuSans}
%\renewcommand*\familydefault{\sfdefault} %% Only if the base font of the document is to be sans serif
\begin{document}
	
	\input{data.tex}	
	
	
	%Det enviroment der bruges til den enkelte opskrift. De forskellige variabler er
	% Arg1 = Opskriftens navn
	% Arg2 = Beskrivelse af opskriften
	% Arg3 = Antal personer
	% Arg4 = Udledning af CO2
	% Arg5 = CO2 rating
	% Arg6 = Tid
	% Arg7 = Ingridiensliste
	% Arg8 = Fremgangsmåde
	\setRecipeColors{numeration=magenta}
	\newenvironment{rurecipe}[8]{
		\begin{recipe}
		[%
			portion={\portion{#3}},
			preparationtime={#6},
			calory={Katagori #5 (#4)}
		]
		{\textcolor{magenta}{#1}}
		
		
		\ingredients
		{%
			#7
		}
		
		\introduction{
			#2
		}
				
		\preparation{
%	        \begin{enumerate}[align=left]
			#8	
%			\end{enumerate}
		}
			
				
%		Estimeret tidsforbrug: #5
		
	\end{recipe}
	}
	{\newpage}
	
	%Det enviroment der bruges til ingridienslister
	\newenvironment{ruingredients}{}{}

	% Den command der bruges til ingridienser. Formen er Navn, mængde, unit
	\newcommand{\ruingredient}[3]{ \unit[#2]{#3} & #1 \\ }

	% Kommando til at beskrive steps
	\newcommand{\rustep}[1]{\step #1}

    % Kommand der kaldes for hver enkelt ingredients.
    % Navn, enhed og coo
	\newcommand{\rucooingredient}[3]{ #1&#2&#3\\\hline}

    % Kommand der kaldes for hver enkelt opskrift.
    % co2, rating og navn.
    \newcommand{\rucoorecipe}[3]{ #1&#2&#3\\\hline}


%%%%%%%%%%%%%%%%%%%%%%%%%%%%%%%%%%%%%%%%%%%%%%%%%%%%%%%%%%%%%%%%%%%%%%%%%%%%%%%%%%%%%%
% Renew the steps command to use RU logo.
\renewcommand{\step}
{
	\lettrine
	[%
	lines=2,
	lhang=0, % space into margin, value between 0 and 1
	loversize=0.15, % enlarges the height of the capital
	slope=0em,
	findent=1em, % gap between capital and intended text
	nindent=0em % shifts all intended lines, begining with the second line
	]{\includegraphics[height=2ex]{ru-logo}}{}%
}

	% Kommando til ny sektion (frokost, morgenmad, osv)
	\newcommand{\rusection}[1]{ \fakesection{#1}}
	
% Opsætning af xcookybooky
\setHeadlines {
	inghead = Ingredienser,
	hinthead = Tip,
	prephead = Tilberedning,
	continuationhead = Forsættes \dots,
	%	calory= Kg \coo ækvivalenter (person/total)
	calory= \coo
}
\renewcommand{\portion}{Antal personer: }

%\setRecipenameFont{pbsi}{T1}{xl}{n}




%%%%%%%%%%%%%%%%%%%%%%%%%%%%%%%%%%%%%%%%%%%%%%%%%%%%%%%%%%%%%%%%%%%%%
%Fake section der tillader at bruge section features uden den irriterende overskrift
\newcommand{\fakesection}[1]{%
	\par\refstepcounter{section}% Increase section counter
	\sectionmark{#1}% Add section mark (header)
	\addcontentsline{toc}{section}{\protect\numberline{\thesection}#1}% Add section to ToC
	% Add more content here, if needed.
}

%%%%%%%%%%%%%%%%%%%%%%%%%%%%%%%%%%%%%%%%%%%%%%%%%%%%%%%%%%%%%%%%%%%%%5
% Selve dok\title{Examples for using \textbf{xcookybooky}}
%opening
\title{FPR's klimavenlige kogebog}
\author{}
\thispagestyle{empty}
\begin{center}
{\textbf{\textcolor{magenta}{\Huge FPR's Klimavenlige Kogebog}}}
\end{center}
\begin{figure}[H]
	\begin{center}
		\includegraphics [width=500pt]{Forside}
	\end{center}
\end{figure}
\newpage

\thispagestyle{empty}
\noindent I vores evige bestræbelser på at bekæmpe klimaforandringer, hvordan kan vi så leve mere klimavenligt? Som almindelige forbrugere kan vi påvirke en hel del med, hvad vi spiser. Men hvor meget påvirker vi egentlig klimaet med en enkel ret. Vi har med denne kogebog sat os for både at finde klimavenlige retter og prøve at udregne, hvor meget de påvirker klimaet. Opgaven er kompliceret, og vores fokus har primært været på udledningen fra produktion og transport af råvarerne i maden, da dette giver det største bidrag.\\

Vi håber, at I vil bruge kogebogen som inspiration til en mere klimavenlig kost – både derhjemme og til arrangementer i Radikal Ungdom, lokalt såvel som nationalt. Vi har lavet kogebogen i tre standardversioner: 4 personer, 10 personer og 80 personer. Dette er udgaven til \ruNumPersons{} personer. Hvis du ønsker en version med opskrifter til et andet antal personer, så kan du kontakte Baltazar Dydensborg på \href{mailto:baltazardydensborg@gmail.com}{baltazardydensborg@gmail.com} og så sender vi jer en kogebog specifikt indrettet til jeres behov.\\

Da hele tanken bag kogebogen er at skåne klimaet, udkommer den kun digitalt. Så kan vi også løbende opdatere kogebogen med nye opskrifter. Vi opfordrer jer derfor kraftigt til IKKE AT PRINTE hele kogebogen. Hvis I skal printe noget, så nøjes med at printe den enkelte opskrift, som I skal bruge.\\

Hvis I har spørgsmål, feedback, forslag til nye opskrifter eller andet input til kogebogen, er I meget velkommen til at kontakte os. Kogebogen inkluderer allerede nu en masse arbejde fra RU'ere der har indsendt opskrifter og forslag, hvilket vi er meget taknemmelige for, og vi sætter altid pris på ris eller ros samt mere kulinarisk inspiration\\

Det skal noteres at denne version er en betaversion - eller sagt på en anden måde: Det er et igangeværende arbejde, og vi håber at i vil hjælpe os med at teste det. Det vil sige at der formentlig vil være fejl, mangler og andre problemer. Det er dog vores håb, at i, når i ser sådanne ting, vil rapportere dem til os, så vi kan gøre kogebogen endnu bedre. Derudover betyder det også, at vi forventer at tilføje flere funktioner senere. Blandt dem forestiller vi os f.eks. mere fokus på årstidssvarende mad, såvel som flere retter til de forskellige katagorier.\\

Held og lykke med madlavningen!\\
Grønne hilsner fra Radikal Ungdoms projektgruppe for den klimavenlige kogeboge\\
Emma Klostergaard Christensen, Nadia Gullestrup Christensen, Helga Leonore Fich Askagaard, Asta Meno, Mette Paaske Sandberg, Philip Tarning-Andersen og Baltazar Dydensborg

\newpage
\thispagestyle{empty}
\section*{Hvordan fungerer kogebogen?}

Kogebogen er inddelt i afsnit efter hvilket måltid de forskellige opskrifter som udgangspunkt er tiltænkt. Der er dog intet der forhindrer brug af aftensmadsretter til frokost, f.eks., hvis man ønsker at lave varm mad til et arrangement. \\

Vi har givet de forskellige retter  en karakter fra 1-5 efter hvor meget \coo de udleder, hvor vurderingen 1 er givet til de mest klimavenlige, imens 5 er givet til de mindst. Dette har vi gjort for at simplificere de udregninger der ligger bag opskrifterne. Derudover står \coo udledningen pr. person og for hele retten også på hver opskrift. Begge dele er regnet i \coo ækvivalenter, en enhed der beskriver hvor mange drivhusgasser der er udledt, omregnet til en fælles enhed.

Derudover er der på hver opskrift forsøgt beskrevet den estimerede tid brugt på at lave opskriften. Det er dog blot et estimat, og vi kan selvfølgelig ikke garantere for at det passer hver gang. Det burde dog give et nogenlunde retvisende billede af tidsforbruget, og hvis det er helt skudt forbi, skal i være velkomne til at skrive til os.

\newpage
\tableofcontents

\vspace{5em}
\newpage

%\section*{Opskrifter}
\input{recipes.tex}
	
\newpage

\section{Ingrediensliste}

Herunder følger en liste over alle de ingredienser der er blevet benyttet i kogebogen - og deres \coo påvirkninger pr. enhed. Det er tanken at i kan bruge den som opslagsværk, såfremt i er nysgerrige på specifikke ingredienser, hvis i vil vurdere hvor meget forskel in- eller eksklusionen af en enkelt ingrediens gør i en opskrift, eller simpelthen som almen opslagsværk i forbindelse med diskussioner om fødevarers klimapåvirkning. Det skal noteres at nogle værdier er fundet ud fra estimater, beregninger og til tider kvalificerede gæt, hvilket selvfølgelig vil sige at vi tager fejl nogle gange. Derudover har vi besluttet at vi ikke regner på klimapåvirkningen af kryderier. Det skyldes deres meget lille mængde udledning, yderligere forsmået af den mængde af dem der bruges i en opskrift. Derfor vil en ingrediens der ikke har en \coo værdi altså typisk være en vi har vurderet til at være et krydderi i en given opskrift. En anden forklarimg kan være at mængden er så lav at den er blevet rundet ned til 0. Overordnet skulle tallene dog være relativt retvisende. Som før er \coo udledningen beskrevet i kg. \coo ækvivalenter pr. enhed.\\
	
	\begin{longtable}{|l|l|l|}
		\hline
\textbf{Navn} & \textbf{Enhed} & \textbf{\coo}\\ \hline
\rucooingredients		
	\end{longtable}

\newpage
\section{Opskriftsliste}

Liste over opskrifter, sorteret efter $CO_2$ belastning.

\begin{longtable}{|l|l|l|}
	\hline
	\textbf{\coo} & \textbf{Karakter} & \textbf{Navn}\\ \hline
	\rucoorecipes		
\end{longtable}

\newpage
\section{Litteratur og bidragsydere}

Herunder følger de forskellige kilder vi har brugt i forbindelse med prodkutionen af kogebogen. Derudover har vi her inkluderet en liste over de RU'ere der har været så flinke at indsende opskrifter, til hvem en kæmpe tak skal lyde

\begin{thebibliography}{9}
	\bibitem{Unilever}
	Unilever Food Solutions\\
	\textit{\coo beregneren}\\
	http://www.unileverfoodsolutions.dk/inspiration-til-dig/your-menu/klimasmart/CO2-beregner\\
	Hentet i Juli 2017
	
	\bibitem{Matopskrifter.no}
	Matopskrifter\\
	\textit{Stor omregningstabell for matvarer}\\
	http://www.matoppskrift.no/sider/Omformer.asp?ID=dl\\
	Hentet i Juli 2017
	
	\bibitem{DTUFOOD}
 Karin Hess Ygil\\
	\textit{Mål, vægt og portionsstørrelser på fødevarer}\\
	Produceret i år 2013
	Hentet i Juli 2017. Kan findes på \url{www.food.dtu.dk}
	
	\bibitem{DJF}
	Lisbeth Mogensen, Ulla Kidmose og John E. Hermansen
	\textit{Fødevarernes klimaaftryk,
		sammenhæng mellem kostpyramiden og klimapyramiden,
		samt omfang og effekt af fødevarespild}
	Udgivet 15 Juni 2009
	
\end{thebibliography}

\newpage

\begin{center}
	\section*{Tak til}
	\textit{Tak til}\\
	
	Bjarke Slater\\
	
	Karoline Bendixen\\
	
	Helena Tjørnelund Christensen\\
	
	Caroline Valentiner-Branth\\
	
	Sigrid Friis Proschowsky\\
	
	\textit{For deres bidrag med opskrifter til kogebogen}
\end{center}



	\newpage
	\centering Radikal Ungdoms Klimavenlige Kogebog - version 0.3\\
	Produceret af arbejdsgruppen for den Klimavenlige Kogebog, bestående af Asta Meno, Emma Klostergaard Christensen, Helga Leonore Fich Askgaard, Mette Paaske Sandberg, Nadia Gullestrup Christensen, Philip Tarning-Andersen og Baltazar Dydensborg\\
	Fejl, problemer og forslag kan rapporteres til Baltazar Dydensborg på \href{mailto:baltazardydensborg@gmail.com}{baltazardydensborg@gmail.com}\\
	Denne version er produceret d. \today, og er beregnet til \ruNumPersons{} personer
\end{document}



