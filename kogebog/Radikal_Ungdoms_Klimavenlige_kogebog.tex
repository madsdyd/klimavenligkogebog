
\documentclass[11pt, a4paper]{article}
\usepackage{geometry}
\usepackage{amsmath, amssymb}
\usepackage{graphicx}
\usepackage[utf8]{inputenc}
%\usepackage{chemformula}
%\usepackage{chemstyle}
%\usepackage{chemmacros}
\usepackage{float}
\usepackage{color}
\definecolor{amethyst}{rgb}{0.6, 0.4, 0.8}
\color{black}
\usepackage[danish]{babel}
\geometry{
	verbose,
	tmargin=2.5cm,
	bmargin=2.5cm,
	lmargin=2.5cm,
	rmargin=2.5cm
}
\setlength{\parindent}{0ex}

\newcommand{\coo}{CO$_2$ }

%opening
\title{Matematik B - Hjemmeopgave til d. 21/4 2017}
\author{af Baltazar Dydensborg}


\begin{document}
	
	%Det enviroment der bruges til den enkelte opskrift. De forskellige variabler er
	% Arg1 = Opskriftens navn
	% Arg2 = Beskrivelse af opskriften
	% Arg3 = Antal personer
	% Arg4 = Udledning af CO2
	% Arg5 = Tid
	% Arg6 = Ingridiensliste
	% Arg7 = Fremgangsmåde
	\newenvironment{recipe}[7]{
		\section*{#1}
		\textit{#2}\\
		
		Antal personer: #3\\
		
		Kg \coo ækvivalenter (pr. person/hele retten): #4\\
		
		Estimeret tidsforbrug: #5
		\subsection*{Ingredienser}
		#6
		\subsection*{Fremgangsmåde}
		#7
	}
	{\newpage}
	
	%Det enviroment der bruges til ingridienslister
	\newenvironment{ingredients}{\begin{itemize}
	}{	\end{itemize}}

	% Den command der bruges til ingridienser. Formen er Navn, mængde, unit
	\newcommand{\ingredient}[3]{
	\item #2 #3 #1
}


\input{recipes.tex}
	
	
	
\end{document}



