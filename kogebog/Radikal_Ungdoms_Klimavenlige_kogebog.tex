
\documentclass[11pt, a4paper]{article}
\usepackage{geometry}
\usepackage{amsmath, amssymb}
\usepackage{graphicx}
\usepackage[utf8]{inputenc}
\usepackage{xcookybooky}
%\usepackage{chemformula}
%\usepackage{chemstyle}
%\usepackage{chemmacros}
\usepackage{float}
\usepackage{color}
\definecolor{amethyst}{rgb}{0.6, 0.4, 0.8}
\color{black}
\usepackage[danish]{babel}
\geometry{
	verbose,
	tmargin=2.5cm,
	bmargin=2.5cm,
	lmargin=2.5cm,
	rmargin=2.5cm
}
\setlength{\parindent}{0ex}

\newcommand{\coo}{CO$_2$ }

%opening
\title{Matematik B - Hjemmeopgave til d. 21/4 2017}
\author{af Baltazar Dydensborg}


\begin{document}
	
	%Det enviroment der bruges til den enkelte opskrift. De forskellige variabler er
	% Arg1 = Opskriftens navn
	% Arg2 = Beskrivelse af opskriften
	% Arg3 = Antal personer
	% Arg4 = Udledning af CO2
	% Arg5 = Tid
	% Arg6 = Ingridiensliste
	% Arg7 = Fremgangsmåde
	\newenvironment{rurecipe}[7]{
		\begin{recipe}
		[%
			portion={\portion{#3}},
			preparationtime={#5},
			calory={#4}
		]
		{#1}
		
		
		\ingredients
		{%
			#6
		}
		
		\introduction{
			#2
		}
				
		\preparation{
			 #7	
		}
			
				
%		Estimeret tidsforbrug: #5
		
	\end{recipe}
	}
	{\newpage}
	
	%Det enviroment der bruges til ingridienslister
	\newenvironment{ruingredients}{}{}

	% Den command der bruges til ingridienser. Formen er Navn, mængde, unit
	\newcommand{\ruingredient}[3]{ #2 #3 & #1 \\ }

	% Kommando til at beskrive steps
	\newcommand{\rustep}[1]{ \step #1}

% Opsætning af sprog af xcookybooky
\setHeadlines {
	inghead = Ingredienser,
	hinthead = Tip,
	prephead = Tilberedning,
	continuationhead = Forsættes \dots,
	calory= Kg \coo ækvivalenter (person/total)
}

\renewcommand{\portion}{Antal personer: }


\input{recipes.tex}
	
	
	
\end{document}



