
\documentclass[11pt, a4paper]{article}
\usepackage{xcolor}
\usepackage{geometry}
\usepackage{amsmath, amssymb}
\usepackage{graphicx}
\usepackage[utf8]{inputenc}
\usepackage{xcookybooky}
\usepackage{chemformula}
\usepackage{chemstyle}
\usepackage{chemmacros}
\usepackage{graphicx}
%\usepackage[export]{adjustbox}
\usepackage{enumitem}

%\usepackage{ pbsi }

\usepackage{float}
\usepackage{color}
\definecolor{magenta}{RGB}{195, 0, 122}
\color{black}
\usepackage[danish]{babel}
\geometry{
	verbose,
	tmargin=2.5cm,
	bmargin=2.5cm,
	lmargin=2.5cm,
	rmargin=2.5cm
}
\setlength{\parindent}{0ex}

\newcommand{\coo}{CO$_2$ }






\usepackage{hyperref}    % must be the last package
\hypersetup{%
	pdfauthor            = {Radikal Ungdom},
	pdftitle             = {Klimavenlig kogebog},
	pdfsubject           = {Madlavning},
	pdfkeywords          = {madlavning, opskrifter, klima},
	pdfstartview         = {FitV},
	pdfview              = {FitH},
	pdfpagemode          = {UseNone}, % Options; UseNone, UseOutlines
	bookmarksopen        = {true},
	pdfpagetransition    = {Glitter},
	colorlinks           = {true},
	linkcolor            = {black},
	urlcolor             = {magenta},
	citecolor            = {black},
	filecolor            = {black},
}

\begin{document}
	
\input{data.tex}	
	
	
	%Det enviroment der bruges til den enkelte opskrift. De forskellige variabler er
	% Arg1 = Opskriftens navn
	% Arg2 = Beskrivelse af opskriften
	% Arg3 = Antal personer
	% Arg4 = Udledning af CO2
	% Arg5 = Tid
	% Arg6 = Ingridiensliste
	% Arg7 = Fremgangsmåde
	\setRecipeColors{numeration=magenta}
	\newenvironment{rurecipe}[7]{
		\begin{recipe}
		[%
			portion={\portion{#3}},
			preparationtime={#5},
			calory={#4 (Kg ækvivalenter per person/total)}
		]
		{\textcolor{magenta}{#1}}
		
		
		\ingredients
		{%
			#6
		}
		
		\introduction{
			#2
		}
				
		\preparation{
%	        \begin{enumerate}[align=left]
			#7	
%			\end{enumerate}
		}
			
				
%		Estimeret tidsforbrug: #5
		
	\end{recipe}
	}
	{\newpage}
	
	%Det enviroment der bruges til ingridienslister
	\newenvironment{ruingredients}{}{}

	% Den command der bruges til ingridienser. Formen er Navn, mængde, unit
	\newcommand{\ruingredient}[3]{ \unit[#2]{#3} & #1 \\ }

	% Kommando til at beskrive steps
	\newcommand{\rustep}[1]{\step #1}

% Renew the steps command to use RU logo.
\renewcommand{\step}
{
	\lettrine
	[%
	lines=2,
	lhang=0, % space into margin, value between 0 and 1
	loversize=0.15, % enlarges the height of the capital
	slope=0em,
	findent=1em, % gap between capital and intended text
	nindent=0em % shifts all intended lines, begining with the second line
	]{\includegraphics[height=2ex]{ru-logo}}{}%
}

	% Kommando til ny sektion (frokost, morgenmad, osv)
	\newcommand{\rusection}[1]{ \section{#1}}
	
% Opsætning af xcookybooky
\setHeadlines {
	inghead = Ingredienser,
	hinthead = Tip,
	prephead = Tilberedning,
	continuationhead = Forsættes \dots,
	%	calory= Kg \coo ækvivalenter (person/total)
	calory= \coo
}
\renewcommand{\portion}{Antal personer: }

%\setRecipenameFont{pbsi}{T1}{xl}{n}






%%%%%%%%%%%%%%%%%%%%%%%%%%%%%%%%%%%%%%%%%%%%%%%%%%%%%%%%%%%%%%%%%%%%%5
% Selve dok\title{Examples for using \textbf{xcookybooky}}
%opening
\title{Radikal Ungdoms klimavenlige kogebog}
\author{}
\maketitle

\begin{abstract}
\noindent 	Radikal Ungdoms arbejdsgruppe for arbejdet med den klimavenlige kogebog præsenterer her Radikal Ungdoms helt egen klimavenlige kogebog - version 0.1. Opskrifterne her er valgt ud grundet deres relativt lave udledning af \coo ækvivalenter per person, en indikation på at de er gode for jordens klima. Derudover vil den høje mængde af grøntsager forbundet med de fleste opskrifter betyde at mange af dem også er ret gode for dit personlige helbred - hvis man altså holder igen med fedtstofferne.\\
	
\noindent	Kogebogen kan som standard fås beregnet til 3 forskellige mængder personer: 4, 10 og 80. Du sidder med \ruNumPersons{}  udgaven. Hvis dit møde ikke passer den størrelse, er du altid velkommen til at skrive til Philip Tarning-Andersen, ordfører for Klima, Energi, Miljø og Infrastruktur, på \href{mailto:philiptarning@radikalungdom.dk}{philiptarning@radikalungdom.dk}, eller til mig. Så sender vi dig en kogebog specifikt indrettet til antallet af deltagere. Kogebogen er, for at spare papiret, som udgangspunkt kun beregnet til elektronisk brug. Vi opfordrer selvfølgelig jer til at vælge klimaet først, og være med på den idé. Hvis i dog alligevel skulle se det nødvenigt at printe kogebogen, anbefaler vi at i kun printer den ene opskrift i har brug for. Vi kan i øvrigt heller ikke garantere for hvordan et sådant udskrift kommer til at se ud.\\
	
\noindent	Vi håber i kan bruge kogebogen. Hvis i skulle have spørgsmål, feedback, foreslag til nye opskrifter, eller andet der vedkommer kogebogsprojektet, skal i være velkommen til at skrive, enten til Philip eller jeg. Vi skal prøve at svare efter bedste evne, men hvis spørgsmålene er af kulinarisk karakter, kan det være vi må se os nødsagede til at sende dem videre til nogen af de mere køkkenkundige medlemmer af arbejdsgruppen.\\

\noindent Til sidst skal lyde en speciel tak til alle dem der har indsendt opskrifter til kogebogen. Selvom det har været noget af en kamp for naturvidenskabeligt funderede individer som Philip og jeg at kæmpe os igennem enheder som "En knivspids" eller "En håndfuld", har det været en stor hjælp med alle de spændende indspark vi aldrig selv var kommet på\\
	
\noindent	Held og lykke med madlavningen!\\
 På vegne af arbejdsgruppen for den klimavenlige kogebog\\
	Baltazar Dydensborg\\
	Ansvarlig for opsætning og beregninger\\
	\href{mailto:baltazardydensborg@gmail.com}{baltazardydensborg@gmail.com}
\end{abstract}
\newpage
\tableofcontents

\vspace{5em}
\newpage

%\section*{Opskrifter}
\input{recipes.tex}
	
\newpage

\rusection{Ingrediensliste}

Herunder følger en liste over alle de ingredienser der er blevet benyttet i kogebogen - og deres \coo påvirkninger pr. enhed. Det er tanken at i kan bruge den som opslagsværk, såfremt i er nysgerrige på specifikke ingredienser, hvis i vil vurdere hvor meget forskel in- eller eksklusionen af en enkelt ingrediens gør i en opskrift, eller simpelthen som almen opslagsværk i forbindelse med diskussioner om fødevarers klimapåvirkning. Det skal noteres at nogle værdier er fundet ud fra estimater, beregninger og til tider kvalificerede gæt, hvilket selvfølgelig vil sige at vi tager fejl nogle gange. Overordnet skulle tallene dog være relativt retvisende\\

INDSÆT TABEL HER

\rusection{Kilder og bidragsydere}

Herunder følger de forskellige kilder vi har brugt i forbindelse med prodkutionen af kogebogen. Derudover har vi her inkluderet en liste over de RU'ere der har været så flinke at indsende opskrifter, til hvem en kæmpe tak skal lyde

\begin{thebibliography}{9}
	\bibitem{Unilever}
	Unilever Food Solutions\\
	\textit{\coo beregneren}\\
	http://www.unileverfoodsolutions.dk/inspiration-til-dig/your-menu/klimasmart/CO2-beregner\\
	Hentet i Juli 2017
	
	\bibitem{Matopskrifter.no}
	Matopskrifter\\
	\textit{Stor omregningstabell for matvarer}\\
	http://www.matoppskrift.no/sider/Omformer.asp?ID=dl\\
	Hentet i Juli 2017
	
	\bibitem{DTUFOOD}
 Karin Hess Ygil\\
	\textit{Mål, vægt og portionsstørrelser på fødevarer}\\
	Produceret i år 2013
	Hentet i Juli 2017. Kan findes på \url{www.food.dtu.dk}
	
	\bibitem{DJF}
	Lisbeth Mogensen, Ulla Kidmose og John E. Hermansen
	\textit{Fødevarernes klimaaftryk,
		sammenhæng mellem kostpyramiden og klimapyramiden,
		samt omfang og effekt af fødevarespild}
	Udgivet 15 Juni 2009
	
\end{thebibliography}
	\newpage
	\centering Radikal Ungdoms Klimavenlige Kogebog - version 0.1\\
	Produceret af arbejdsgruppen for den Klimavenlige Kogebog, bestående af Asta Meno, Emma Klostergaard Christensen, Helga Leonore Fich Askgaard, Mette Paaske Sandberg, Nadia Gullestrup Christensen, Philip Tarning-Andersen og Baltazar Dydensborg\\
	Denne version er produceret d. \today, og er udregnet efter INDSÆT ANTAL MENNESKER HER
\end{document}



