
\documentclass[11pt, a4paper]{article}
% Set up our encodings.
% The order of the next two matters.
% \usepackage[T1]{fontenc}
\usepackage[utf8]{inputenc}
\usepackage{textcomp}

\usepackage{xcolor}
\usepackage{geometry}
\usepackage{amsmath, amssymb}
\usepackage{graphicx}

\usepackage{xcookybooky}
\usepackage{graphicx}
\usepackage{longtable}
%\usepackage[export]{adjustbox}
\usepackage{enumitem}

%\usepackage{ pbsi }

\usepackage{float}
\usepackage{color}


%\definecolor{magenta}{RGB}{195, 0, 122}
\definecolor{magenta}{RGB}{70, 169, 0}
\color{black}
\usepackage[danish]{babel}
\geometry{
	verbose,
	tmargin=2.5cm,
	bmargin=2.5cm,
	lmargin=2.5cm,
	rmargin=2.5cm
}
\setlength{\parindent}{0ex}

\newcommand{\coo}{CO$_2$ }

\usepackage{hyperref}    % must be the last package
\hypersetup{%
	pdfauthor            = {Miljørådet FPR},
	pdftitle             = {Klimavenlig kogebog},
	pdfsubject           = {Madlavning},
	pdfkeywords          = {madlavning, opskrifter, klima},
	pdfstartview         = {FitV},
	pdfview              = {FitH},
	pdfpagemode          = {UseNone}, % Options; UseNone, UseOutlines
	bookmarksopen        = {true},
	pdfpagetransition    = {Glitter},
	colorlinks           = {true},
	linkcolor            = {black},
	urlcolor             = {magenta},
	citecolor            = {black},
	filecolor            = {black},
}

%%%%%%%%%%%%%%%%%%%%%%%%%%%%%%%%%%%%%%%%%%%%%%%%%%%%%%%%%%%%%%%%%%%%%%%%%%%%%%%%%%%%%%
% Use dejavu sans font
%\usepackage{DejaVuSans}
%\renewcommand*\familydefault{\sfdefault} %% Only if the base font of the document is to be sans serif
\begin{document}
	
	\input{data.tex}	
	
	
	%Det enviroment der bruges til den enkelte opskrift. De forskellige variabler er
	% Arg1 = Opskriftens navn
	% Arg2 = Beskrivelse af opskriften
	% Arg3 = Antal personer
	% Arg4 = Udledning af CO2
	% Arg5 = CO2 rating
	% Arg6 = Tid
	% Arg7 = Ingridiensliste
	% Arg8 = Fremgangsmåde
	\setRecipeColors{numeration=magenta}
	\newenvironment{rurecipe}[8]{
		\begin{recipe}
		[%
			portion={\portion{#3}},
			preparationtime={#6},
			calory={Katagori #5 (#4)}
		]
		{\textcolor{magenta}{#1}}
		
		
		\ingredients
		{%
			#7
		}
		
		\introduction{
			#2
		}
				
		\preparation{
%	        \begin{enumerate}[align=left]
			#8	
%			\end{enumerate}
		}
			
				
%		Estimeret tidsforbrug: #5
		
	\end{recipe}
	}
	{\newpage}
	
	%Det enviroment der bruges til ingridienslister
	\newenvironment{ruingredients}{}{}

	% Den command der bruges til ingridienser. Formen er Navn, mængde, unit
	\newcommand{\ruingredient}[3]{ \unit[#2]{#3} & #1 \\ }

	% Kommando til at beskrive steps
	\newcommand{\rustep}[1]{\step #1}

    % Kommand der kaldes for hver enkelt ingredients.
    % Navn, enhed og coo
	\newcommand{\rucooingredient}[3]{ #1&#2&#3\\\hline}

    % Kommand der kaldes for hver enkelt opskrift.
    % co2, rating og navn.
    \newcommand{\rucoorecipe}[3]{ #1&#2&#3\\\hline}


%%%%%%%%%%%%%%%%%%%%%%%%%%%%%%%%%%%%%%%%%%%%%%%%%%%%%%%%%%%%%%%%%%%%%%%%%%%%%%%%%%%%%%
% Renew the steps command to use RU logo.
\renewcommand{\step}
{
	\lettrine
	[%
	lines=2,
	lhang=0, % space into margin, value between 0 and 1
	loversize=0.15, % enlarges the height of the capital
	slope=0em,
	findent=1em, % gap between capital and intended text
	nindent=0em % shifts all intended lines, begining with the second line
	]{\includegraphics[height=2ex]{ru-logo}}{}%
}

	% Kommando til ny sektion (frokost, morgenmad, osv)
	\newcommand{\rusection}[1]{ \fakesection{#1}}
	
% Opsætning af xcookybooky
\setHeadlines {
	inghead = Ingredienser,
	hinthead = Tip,
	prephead = Tilberedning,
	continuationhead = Forsættes \dots,
	%	calory= Kg \coo ækvivalenter (person/total)
	calory= \coo
}
\renewcommand{\portion}{Antal personer: }

%\setRecipenameFont{pbsi}{T1}{xl}{n}




%%%%%%%%%%%%%%%%%%%%%%%%%%%%%%%%%%%%%%%%%%%%%%%%%%%%%%%%%%%%%%%%%%%%%
%Fake section der tillader at bruge section features uden den irriterende overskrift
\newcommand{\fakesection}[1]{%
	\par\refstepcounter{section}% Increase section counter
	\sectionmark{#1}% Add section mark (header)
	\addcontentsline{toc}{section}{\protect\numberline{\thesection}#1}% Add section to ToC
	% Add more content here, if needed.
}

%%%%%%%%%%%%%%%%%%%%%%%%%%%%%%%%%%%%%%%%%%%%%%%%%%%%%%%%%%%%%%%%%%%%%5
% Selve dok\title{Examples for using \textbf{xcookybooky}}
%opening
\thispagestyle{empty}
\section*{Introduktion}

Som mennesker i Danmark, og andre dele af verden, har vi et stort overforbrug som vi selv kan være med til at nedskære på. Et overforbrug betyder at vi køber mere end vi har brug for, det vil også sige, at nu mere vi forbruger, nu mere CO2 bliver der udledt, bl.a. transport og produktion. Heldigvis skal der ikke så meget til at ændre sine vaner, og nedskære på forbrug. \\

Det kan gøres på mange måder, og i denne klimavenlige kogebog er der lagt fokus på det vi spiser - altså vores madforbrug. Mad er en nødvendig kilde, så derfor kan det lyde underligt at skære ned på det. Det der menes med at nedskære vores madforbrug, vil sige vi kigger på hver ret, og overvejer om den er klimavenlig, eller ser på de enkle ingredienser, og overvejer om de kan erstattes. Et eksempel kan være man elsker kød, og spiser kød hver dag. Man kan overveje om det er klimavenligt, da kød, især oksekød og lammekød udleder store mængder CO2. Man derfor nedskære sit kødforbrug, til ikke at spise kød hver dag, men hver anden.\\

Denne proces er langvarig, dvs. den går fra at du planlægger hvad du skal spise, til du køber maden, til du spiser den. Når du planlægger hvad du fx skal have til aftensmad, kan du finde inspiration i denne kogebog på enkle klimavenlige opskrifter, samt finde deres klimapåvirkning. Der er beskrevet hvornår ingredienserne, vi anbefaler at der bliver købt danske varer, men hvis varen ikke er i sæson, vil transporten være mere klimavenlig end dyrkning i drivhus. Hvis der er tvivl, kan det nemt findes på internettet, ved eksempelvis slå op hvornår danske grøntsager i sæson eller slå op bagerst i denne kogebog hvor vi har lavet en liste over grøntsager og deres periode. Sæsonvare kan bruges hvis der fx er en dominerende grøntsag i en ret, som enten skal erstattes, eller bare skal vide om det passer til årstiden.
I denne kogebog er retterne beregnet til \ruNumPersons{} personer, så den både kan bruges i selskaber og derhjemme. Vi håber at I vil bruge den kogebog, og skabe inspiration til et mere klimavenligt madforbrug, der vil hjælpe med at blive opmærksomme på vores individuelle CO2 aftryk. 
God fornøjelse med at lave klimavenlig mad, der bære os mod en bæredygtig fremtid!\\



\newpage
\thispagestyle{empty}
\section*{Hvordan fungerer kogebogen?}

Kogebogen er inddelt i afsnit efter hvilket måltid de forskellige opskrifter er tiltænkt, men de forskellige retter kan sagtens laves som man vil mht. undervisningen.\\

Vi har givet de forskellige retter en karakter fra 1-5 efter hvor meget CO2 de udleder, hvor 1 er givet til de mest klimavenlige, mens 5 er givet til de mindst. Dette har vi gjort for at gøre det mere simpel i de udregninger der ligger bag opskrifterne. Derudover står CO2 udledningen pr. person og for hele retten også på hver opskrift. Begge dele er regnet i CO2 i den samme enhed, så man kan se forskellen.\\

Derudover er der på hver opskrift beskrevet den estimerede tid brugt på at lave opskriften. Det er dog blot et estimat, og vi kan selvfølgelig ikke garantere for at det passer hver gang. Det burde give et nogenlunde billede af tidsforbruget.\\


\newpage
\tableofcontents

\vspace{5em}
\newpage

%\section*{Opskrifter}
\input{recipes.tex}
	
\newpage

\section{Ingrediensliste}


Herunder er en liste over alle de ingredienser der er blevet benyttet i kogebogen - og deres \coo påvirkninger pr. enhed. Det er tanken at i kan bruge den som opslagsværk, når I har brug for at undersøge ingredienserne nærmere. Det skal noteres at nogle værdier er fundet ud fra estimater og beregninger, hvilket kan være forkert, men er meget tæt på. Derudover har vi besluttet at vi ikke regner på klimapåvirkningen af kryderier. Da deres meget lille mængde udledning, da de også er mindsket i opsrifterne. Derfor vil en ingrediens der ikke har en CO2 værdi altså typisk være en vi har vurderet til at være et krydderi. En anden forklarimg kan være at mængden er så lav at den er blevet rundet ned til 0. Overordnet skulle tallene være relativt rigtige. Som før er \coo udledningen beskrevet i kg. \coo pr. enhed.



	\begin{longtable}{|l|l|l|}
		\hline
\textbf{Navn} & \textbf{Enhed} & \textbf{\coo}\\ \hline
\rucooingredients		
	\end{longtable}

\newpage
\section{Opskriftsliste}

Liste over opskrifter, sorteret efter $CO_2$ belastning.

\begin{longtable}{|l|l|l|}
	\hline
	\textbf{\coo} & \textbf{Karakter} & \textbf{Navn}\\ \hline
	\rucoorecipes		
\end{longtable}

\newpage
\section{Danske grønsager i sæson}

\begin{longtable}{|l|p{10cm}|}
	\hline
	\textbf{Sæson} & \textbf{Grønsager}\\ \hline
    Vinter & Basilikum, beder, balida (æbler), champignon, citronmelisse, dild, elstar (æbler), glaskål, grønkål, gulerødder, hvidkål, løg, mynte, oregano, pastinak, porre, rosenkål, rosmarin, rødbede, rødkål, rødløg, kinakål, koriander\\ \hline
    Forår & Agurk, beder, basilikum, champignon, dild, elestar, balida, gulerod, løg, mynte, oregano, pastinak, porre, purløg, rødbeder, radiser, rosmarin, skalotteløg, timian, asparges, babyspinat, blommetomat, cherrytomater, bøftomat, rucola, spinat, chili, peberfrugt\\ \hline
    Sommer & Agurk, asparges, babyspinat, bladselleri, blomkål, blommetomat, broccoli, champignon, chili, citronmelisse, cocktailtomater, dild, iceberg salat, sommer hindbær, jordbær,  løg, peberfrugt, persille, porre, rabarber, rocula, rødløg, spidskål, spinat, sukkerærter, ærter, stikkelsbær, blåbær, bønner, kirsebær, majs, solbær, squash, blommer, brombær, guldborg (æbler)\\ \hline
    Efterår & Agurk, babyspinat, beder, blomkål, blommetomat, bladselleri, blommer, blåbær, brombær, bøftomat, bønner, champingnon, cherrytomat, chili, citronmelisse, forårsløg, glaskål, gulerødder, hindbær, hvidkål, iceberg, løg, majs, pastinak, peberfrugt, porre, purløg, radiser, æbler, rødbede, rødkål, skalotteløg, broccoli, rødløg, squash\\ \hline
\end{longtable}


\newpage
\section{Kildeliste}

\begin{thebibliography}{9}
	\bibitem{Unilever}
	Unilever Food Solutions\\
	\textit{\coo beregneren}\\
	http://www.unileverfoodsolutions.dk/inspiration-til-dig/your-menu/klimasmart/CO2-beregner\\
	Hentet i Juli 2017
	
	\bibitem{Matopskrifter.no}
	Matopskrifter\\
	\textit{Stor omregningstabell for matvarer}\\
	http://www.matoppskrift.no/sider/Omformer.asp?ID=dl\\
	Hentet i Juli 2017
	
	\bibitem{DTUFOOD}
 Karin Hess Ygil\\
	\textit{Mål, vægt og portionsstørrelser på fødevarer}\\
	Produceret i år 2013
	Hentet i Juli 2017. Kan findes på \url{www.food.dtu.dk}
	
	\bibitem{DJF}
	Lisbeth Mogensen, Ulla Kidmose og John E. Hermansen
	\textit{Fødevarernes klimaaftryk,
		sammenhæng mellem kostpyramiden og klimapyramiden,
		samt omfang og effekt af fødevarespild}
	Udgivet 15 Juni 2009
	
	\bibitem{FIS}
	https://saeson-web.dk/frugt-groent-i-saeson/
	
	
\end{thebibliography}

\newpage

\end{document}



